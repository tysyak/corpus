\documentclass{IEEEtran}

\usepackage[letterpaper]{geometry}
\usepackage{minted}
\usepackage[spanish]{babel}

\begin{document}
\title{Introducción al TFIDF}
\author{Romero Andrade Cristian}
\markboth{24~de~marzo~de~2021}{}
\maketitle{}

\begin{abstract}
  Un abstract
\end{abstract}

\section{Introducción}\label{sec:introduccion}
\IEEEPARstart{L}{a} medida Tf-idf  es una medida estadística que se utiliza
en la recuperación de información para evaluar la relevancia de los términos en
los documentos de una colección de documentos.

\begin{itemize}
  \item El tf-idf es el producto de dos estadísticas, frecuencia de término y frecuencia de documento inversa
        Hay varias formas de determinar los valores exactos de ambas estadísticas.
  \item Una fórmula que tiene como objetivo definir la importancia de una palabra clave o frase dentro de un documento o una página web.
\end{itemize}

\subsection{Frecuencia de término}\label{sec:frec-de-term}
La frecuencia del término, $tf(t, d)$, es la frecuencia del término $t$.
\[ tf(t,d)\frac{f_{t,d}}{\sum_{t^{i}\in d}f_{t^{i}, d}}\]
donde $f_{t, d}$ es el recuento bruto de un término en un documento, es decir,
el número de veces que el término $t$ aparece en el documento $d$.
Hay varias otras formas de definir la frecuencia de los términos:
\begin{itemize}
  \item el recuento bruto en sí mismo: $tf(t, d) = f(t, d)$.
  \item ``Frecuencias'' booleanas: $tf(t, d ) = 1$ si $t$ ocurre en $d$ y $0$ en caso
        contrario.
  \item frecuencia de término ajustada a la longitud del documento:
        $tf(t, d) = \frac{f_{t, d}}{(numero\ de\ palabras\ en\ d)}$.
  \item frecuencia escalada logarítmicamente: $tf(t, d) = \log{(1 + f_{ t , d})}$.
  \item frecuencia aumentada, para evitar un sesgo hacia documentos más largos,
        por ejemplo, frecuencia sin procesar dividida por la frecuencia sin
        procesar del término más frecuente en el documento.
        \[ tf(t,d)=0.5+0.5 \times \frac{f_{t, d}}{max\{f_{t^{i}, d}: t^{i} \in d\}} \]
\end{itemize}

\subsection{Frecuencia de documento inversa}\label{sec:frec-de-docum}
La frecuencia inversa del documento es una medida de cuánta información
proporciona la palabra, es decir, si es común o rara en todos los documentos.
Es la fracción inversa escalada logarítmicamente de los documentos que
contienen la palabra (obtenida dividiendo el número total de documentos
por el número de documentos que contienen el término, y luego tomando el
logaritmo de ese cociente):
\begin{itemize}
  \item $N$: Número total de documentos en el corpus $N = |D| $.
  \item $|\{d \in D : t \in d\}|$: número de documentos donde aparece el término,
        es decir, si el término no está en el corpus, esto dará lugar a una división por cero. Por lo tanto, es     común ajustar el denominador a $tf(t,d) \ne 01 + |\{d \in D : t \in d \}|$

\end{itemize}

\subsection{Frecuencia de término: frecuencia de documento inversa}\label{sec:frec-de-term-1}
Entonces tf – idf se calcula como:
\[ tf \times idf(t,d,D) = tf(t,d) \cdot idf(t,D)\]

\section{Desarrollo}\label{sec:desarrollo}




\end{document}
